\documentclass{article}
\usepackage{tikz, incgraph, hyperref, ocgx, chemfig}
\igrset{border=4cm}
\usetikzlibrary{positioning, calc, ocgx, decorations.text, arrows.meta}
\setchemfig{atom sep=20pt, cram width=3pt}
\tikzset{
    default/.append style={
            rectangle,
            rounded corners,
            minimum width=2cm,
            minimum height=1cm
        },
    arrow/.append style={
            -latex,
            shorten >=5pt,
            shorten <=5pt
        },
    line/.append style={
            -,
            shorten >=5pt,
            shorten <=5pt
        },
    deco/.style n args={1}{
            postaction={
                    decorate,
                    decoration={
                            text along path,
                            text align=center,
                            raise=4pt,
                            reverse path,
                            text={#1}
                        }
                }
        }
}
\newcommand{\basic}[2]{
    \begin{ocg}{#1}{#1}{0}
        \textcolor{red}{\hspace{2.5pt}#2}
    \end{ocg}
}
\newcommand{\basicr}[2]{
    \begin{ocg}{#1}{#1}{1}
        \textcolor{red}{\hspace{2pt}#2}
    \end{ocg}
}
\newcommand{\question}[4]{
    \node[default, draw, #1, switch ocg={#2 #2e}](#2){
        \basic{#2}{#3}
    };
    \node at(#2){
        \basicr{#2e}{#4}
    };
}
\newcommand{\comment}[4]{
    \node[#1,
        switch ocg={#2 #2e}](#2){
        \basic{#2}{#3}
    }
    node[#1]{
        \basicr{#2e}{#4}
    };
}
\newcommand{\chem}[3]{
    \node[#1](#2){} node at(#2){\setchemfig{cram width=3pt}\chemfig{#3}};
}