\node[default, draw](dna){DNA};
\question{below=of dna}{rna}{RNA}{?}
\question{left=3cm of dna}{mrna}{mRNA}{?}
\question{right=3cm of dna}{ncRna}{ncRNA}{?}
\draw[arrow](dna) to coordinate(rnaToMrna) (mrna);
\comment{above=0 of rnaToMrna}{transcription}{Transcription}{?}
\draw[arrow](dna) to coordinate(rnaToNcRna) (ncRna);
\comment{above=0 of rnaToNcRna}{ncRnaText}{Transcription}{?}
\comment{left=.2 of mrna}{mrnaText}{$1\sim 3$ \textcolor{black}{\% of total RNA samples}}{? \textcolor{black}{\% of total RNA samples}}
\question{below left=of mrna, xshift=1.7cm}{untranslatedRegion}{Untranslated Regions}{?}
\question{below right=of mrna, xshift=-1.7cm}{codingRegion}{Coding Regions}{?}
\draw[line](mrna) to (untranslatedRegion.north);
\draw[line](mrna) to (codingRegion.north);
\comment{below=of codingRegion}{template}{Act as a template for protein synthesis \textcolor{blue}{(Translation)}}{?}
\draw[line](codingRegion) to (template);
\comment{right=of ncRna}{ncRnaTextSecond}{Not Translated}{?}
\draw[line](ncRna) to (ncRnaTextSecond);
\draw[dashed, brown]($(rna.south)+(0,-.5)$)
-| coordinate(nucleicAcid) ($(dna.east)+(.5,0)$) |- ($(dna.north)+(0,.5)$) -| ($(rna.west)+(-.5,0)$) |- cycle;
\question{right=of nucleicAcid}{nucleicAcidText}{Nucleic Acid}{?}
\comment{above=of nucleicAcidText}{nucleicAcidTextSecond}{Polymer}{?}
\draw[line](nucleicAcidText) to (nucleicAcidTextSecond);
\draw[line, dashed, brown](nucleicAcid) to (nucleicAcidText);
\question{right=of nucleicAcidText}{nucleotide}{Nucleotide}{?}
\comment{below=0 of nucleotide}{nucleotideTextFirst}{\textcolor{black}{1. }Provide Chemical Energy}{\textcolor{black}{1. }?}
\comment{below=0 of nucleotideTextFirst}{nucleotideTextSecond}{\textcolor{black}{2. }Participate in Cell Signaling}{\textcolor{black}{2. }?}
\comment{above=of nucleotide}{nucleotideText}{Monomer}{?}
\draw[line](nucleotide) to (nucleotideText);
\node at($(nucleicAcidText.east)!.5!(nucleotide.west)$){\huge$\ni$};
\node[right=of nucleotide](nucleotideStructure)
{
\chemfig{
O^{-}-P(=[:90]O)(-[:-90]O^{-})-O
-[:-30]-[:-90]([:-18]<[:-60](-[6]OH)-[0,,,,line width=3pt]>[:60]
(@{hydrogen5}-[:90]N@{nitrogen1}
*5([:72]-
*6(-N=-N=(-NH_2)-=)
--N=-))
-[:150,1.15]O-[:210,1.15])
}};
\draw[line](nucleotide) to (nucleotideStructure);
\comment{right=0 of nucleotideStructure}{nucleotideStructureTextSecond}
{\textcolor{black}{Center: }\textcolor{blue}{Ribose} \textcolor{black}{or} Deoxyribose}{\textcolor{black}{Center: }?}
\comment{above=of nucleotideStructureTextSecond.west, anchor=west}{nucleotideStructureTextFirst}
{\textcolor{black}{Upper Left: }Phosphate Group}{\textcolor{black}{Upper Left: }?}
\comment{below=of nucleotideStructureTextSecond.west, anchor=west}{nucleotideStructureTextThird}
{\textcolor{black}{Upper Right: }Nitrogenous Base}{\textcolor{black}{Upper Right: }?}

\draw[dashed, brown](nucleotideStructureTextSecond.north)
-| (nucleotideStructureTextThird.south west)
-| (nucleotideStructureTextThird.east)
|- coordinate(nucleoside) cycle;
\question{right=of nucleoside}{nucleosideText}{Nucleoside}{?}
\draw[dashed, brown](nucleoside) to (nucleosideText);

\question{above=of transcription}{polymerase}{RNA Polymerase}{?}
\draw[line](transcription) to (polymerase);
\question{above=of polymerase}{transcriptionFactor}{Transcription Factor}{?}
\draw[line](polymerase) to node[circle, fill=white]{+} (transcriptionFactor);
\comment{right=of polymerase}{promotor}{Bind to the Promotor}{?}
\draw[arrow](polymerase) to (promotor);
\comment{right=of promotor}{transcriptionBubble}{Generate a Transcription Bubble}{?}
\draw[arrow](promotor) to (transcriptionBubble);
\comment{above=of transcriptionBubble}{transcriptionBubbleText}{\textcolor{black}{Size: }$12\sim 14$ base pairs}{\textcolor{black}{Size: }?}
\draw[line](transcriptionBubble) to (transcriptionBubbleText);
\coordinate(base) at($(nucleotideStructureTextThird.south)+(-1,-.5)$);
\question {below=0 of base}{guanine}{Guanine}{?}
\question{left=.5 of guanine}{adenine}{Adenine}{?}
\question{right=.5 of guanine}{cytosine}{Cytosine}{?}
\question{right=.5 of cytosine}{thymine}{Thymine}{?}
\draw[dashed, green]($(cytosine.west)+(-.2,0)$)
|- ($(thymine.south)+(0,-.2)$)
-| ($(thymine.east)+(.2,0)$)
|- coordinate(pyrimidine) ($(cytosine.north)+(0,.2)$) -| cycle;
\comment{right=of pyrimidine}{pyrimidineText}{Pyrimidine}{?}
\draw[dashed, green](pyrimidine) to (pyrimidineText);
\draw[dashed, blue]($(adenine.west)+(-.2,0)$)
|- coordinate(purine) ($(guanine.south)+(0,-.2)$)
-| ($(guanine.east)+(.2,0)$) |- ($(adenine.north)+(0,.2)$) -| cycle;
\comment{left=of purine}{purineText}{Purine}{?}
\draw[dashed, blue](purine) to (purineText);
\comment{right=of thymine}{uracil}{Replaced with Uracil in RNA}{?}
\draw[-](thymine) to (uracil);
\draw[arrow, latex-latex, out=300, in=240](guanine) to node[below]{pair} (cytosine);
\draw[arrow, latex-latex, out=315, in=225](adenine) to coordinate(adenineToThymine) (thymine);
\draw(nucleotideStructureTextFirst.east) -- ++(.3,0) coordinate(connection) |- (nucleotideStructureTextSecond.east);
\comment{above right=.5 of connection}{connectionText}{Connected to form the long stand of DNA}{?}
\node[right=of connectionText](phosphodiesterBondStructure){
\chemfig{
O=[:-30]P(-[:60]O-[:60])(-[:240]O^{-})-[:-30]O-[:-30]C@{oxygen2}H_2
-[:-90]?[a]@{hydrogen2}<[:-60]@{hydrogen3}
(-[:240]@{oxygen3}O-[:240]P(=[:150]O)(-[:240]O^{-})-[:-30]O-[:-30])
-[0,,,,line width=3pt]
>[:60]-[:150,1.18]O?[a]
}};
\draw[-](connection) to (connectionText.south west);
\draw[-latex](connectionText) to (phosphodiesterBondStructure);
\comment{right=of phosphodiesterBondStructure}{phosphodiesterBond}{Phosphodiester Bond}{?}
\draw[line](phosphodiesterBondStructure) to (phosphodiesterBond);

\coordinate(center3) at($(adenineToThymine)+(0,-2)$);
\coordinate(center4) at($(center3)+(-.5,0)$);
\coordinate(center5) at($(center4)+(1.5cm,.5cm)$);
\draw[yellow, samples=100, shift=(center4)] plot(\x, {cos(deg(2*\x))});
\draw[brown, samples=100, shift=(center5)] plot (\x, {cos(2*deg(\x))});

\draw[dashed, blue]($(center4)+(0,1)$) to ($(center4)+(0,-.5)$) coordinate(minorGroove);
\comment{right=8 of minorGroove}{polynucleotide}{Polynucleotide}{?}
\draw[-]($(polynucleotide.west)+(-1,0)$) to (polynucleotide);
\comment{below=0 of minorGroove}{minorGrooveText}{\textcolor{blue}{Minor Groove}}{?}
\draw[dashed, red]($(center4)+(-1.55,1.5)$) to ($(center4)+(-1.55, -1)$) coordinate(majorGroove);
\comment{below=0 of majorGroove}{majorGrooveText}{\textcolor{red}{Major Groove}}{?}
\comment{below=0 of majorGrooveText}{majorGrooveWidth}{22\textcolor{black}{\AA}}{?\textcolor{black}{\AA}}
\comment{below=0 of minorGrooveText}{minorGrooveWidth}{\textcolor{blue}{12}\textcolor{black}{\AA}}{?\textcolor{black}{\AA}}

\coordinate (center) at($(center4)+(-5.5,-6)$);
\draw[arrow, latex-latex, dashed, red] ($(center)+(-30:4)$) arc (-30:210:4cm);
\draw[arrow, latex-latex, dashed, blue]($(center)+(210:4)$) arc (210:330:4cm);

\node at(center)(adenineThymine){
\chemfig{
[:45]-N*5(-*6(-N=-N?[a]-(-N(-[:120]H)-[:30]H
(-[:0,2,,,thick,dotted,red]O=_[:-60]*6(-N(-H?[a,,thick,dotted,red])-(=_O)-N(-R)-=(-)-))
)-=)--N=-)
}};
\coordinate(center2) at($(center4)+(6.5,-6)$);
\draw[arrow, latex-latex, dashed, red]($(center2)+(-30:4)$) arc (-30:210:4cm);
\draw[arrow, latex-latex, dashed, blue]($(center2)+(210:4)$) arc (210:330:4cm);
\node at(center2)(guanineCytosine){
\chemfig{
[:45]-N*5(-*6(-N=
(-N(-[:240]H)-[:0]@{hydrogen1}H?[b])
-N(-H?[a])=(=O
-[:0,2,,,thick,dotted,red]H-[:0]N(-[:60]H)-[:300]*6(
=N?[a,,thick,dotted,red]-(=_@{oxygen1}O?[b,,thick,dotted,red])-N(-)-=-
)
)-=)--N=-)
}
};
\comment{left=2 of adenineThymine}{adenine}{Adenine}{?}
\draw[-](adenineThymine) to (adenine);
\comment{below right=of adenineThymine}{thymine}{Thymine}{?}
\draw[-](adenineThymine) to (thymine.north west);
\comment{left=2 of guanineCytosine}{guanine}{Guanine}{?}
\draw[-](guanineCytosine) to (guanine);
\comment{right=2 of guanineCytosine}{cytosine}{Cytosine}{?}
\draw[-](guanineCytosine) to (cytosine);
\draw[-latex, dashed]($(center)+(-7,-3)$) coordinate(threePrime2) to ($(center)+(-7,3)$) coordinate(fivePrime2);
\draw[latex-, dashed]($(center2)+(7,-3)$) coordinate(fivePrime3) to ($(center2)+(7,3)$) coordinate(threePrime3);
\comment{right=0 of threePrime2}{threePrimeText}{$3^{\prime}$}{?}
\node[right=0 of fivePrime2](fivePrimeText){$5^{\prime}$};
\comment{right=0 of threePrime3}{threePrimeTextSecond}{$3^{\prime}$}{?}
\comment{right=of threePrimeTextSecond}{hydroxylGroup}{Hydroxyl Group}{?}
\draw[line](threePrimeTextSecond) to (hydroxylGroup);
\comment{right=0 of fivePrime3}{fivePrimeTextSecond}{$5^{\prime}$}{?}
\comment{right=of fivePrimeTextSecond}{phosphateGroup}{Phosphate Group}{?}
\draw[line](fivePrimeTextSecond) to (phosphateGroup);
\node at($(center2)+(0,-4.5)$){\href{https://www.labxchange.org/library/pathway/lx-pathway:c9f00da6-1b2b-43ec-9a0d-de38ccc3f97f/items/lx-pb:c9f00da6-1b2b-43ec-9a0d-de38ccc3f97f:html:e3d9ae3f}{Adress}};
https://www.ncbi.nlm.nih.gov/pmc/articles/PMC3195647/

\node[left=5 of untranslatedRegion](dnaFirst){$5^{\prime}$ ...ATGGGGCTC... $3^{\prime}$};
\comment{right=of dnaFirst}{senseStrand}{Sense Strand}{?}
\draw[line](dnaFirst) to (senseStrand);
\node[below=.25 of dnaFirst](dnaSecond){$3^{\prime}$ ...TACCCCGAG... $5^{\prime}$};
\commentAt{$(dnaFirst.west)!.5!(dnaSecond.west)+(-.5,0)$}{dnaLeft}{DNA}{?}
\comment{right=of dnaSecond}{antisenseStrand}{Antisense Strand}{?}
\draw[line](dnaSecond) to (antisenseStrand);
\comment{below=.5 of antisenseStrand.west, anchor=west}{templateStrand}{Template Strand}{?}
\node[below=of dnaSecond](mrnaFirst){$5^{\prime}$ ...AUGGGGCUC... $3^{\prime}$};
\comment{right=of mrnaFirst}{mrnaWithCodons}{mRNA with codons}{?}
\draw[line](mrnaFirst) to (mrnaWithCodons);
\node[below=.25 of mrnaFirst](mrnaSecond){UACCCCGAG};
\commentAt{$(mrnaFirst.west)+(-.5,-.5)$}{mrnaLeft}{mRNA}{?}
\draw[arrow](dnaSecond) to (mrnaFirst);
\question{below=of mrnaSecond}{trna}{tRNA}{?}
\comment{below=of trna}{peptide}{Peptides}{?}
\comment{right=of peptide}{peptideText}{Short chain of amino acids}{?}
\draw[line](peptide) to (peptideText);
\node[below=of peptide]{\href{https://www.youtube.com/watch?v=-L0rikDOX5s}{Adress}};
\draw[arrow](mrnaSecond) to coordinate(mrnaToTrna) (trna);
\draw[arrow](trna) to coordinate(trnaToPeptide) (peptide);
\comment{below=3.25 of mrnaLeft}{proteinsText}{Protein}{?}
\comment{left=of proteinsText}{proteinsTextText}{Long chain of amino acids}{?}
\draw[line](proteinsText) to (proteinsTextText);



https://www.youtube.com/watch?v=0E4p34mqJbg
% https://www.vedantu.com/question-answer/which-of-the-following-would-not-be-considered-class-12-biology-cbse-5fb37455b8881b0c51d41bf3
% https://www.youtube.com/watch?v=ad6h4ErsSTY

