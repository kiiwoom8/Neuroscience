\node[default, draw](dna){DNA};
\question{below=of dna}{rna}{RNA}{?}
\question{right=3cm of dna}{mrna}{mRNA}{?}
\question{left=3cm of dna}{ncRna}{ncRNA}{?}
\draw[arrow](dna) to coordinate(rnaToMrna) (mrna);
\comment{above=0 of rnaToMrna}{transcription}{Transcription}{?}
\draw[arrow](dna) to coordinate(rnaToNcRna) (ncRna);
\comment{above=0 of rnaToNcRna}{ncRnaText}{Transcription}{?}
\comment{right=.2 of mrna}{mrnaText}{$1\sim 3$ \textcolor{black}{\% of total RNA samples}}{? \textcolor{black}{\% of total RNA samples}}
\comment{above=.5 of mrnaText.west, anchor=west}{mrnaTextFirst}{Coding RNA}{?}
\comment{left=of ncRna}{ncRnaTextSecond}{Not Translated}{?}
\draw[line](ncRna) to (ncRnaTextSecond);
\draw[dashed, brown]($(rna.south)+(0,-.5)$)
-| coordinate(nucleicAcid) ($(dna.east)+(.5,0)$) |- ($(dna.north)+(0,.5)$) -| ($(rna.west)+(-.5,0)$) |- cycle;
\question{right=of nucleicAcid}{nucleicAcidText}{Nucleic Acid}{?}
\comment{above=of nucleicAcidText}{nucleicAcidTextSecond}{Polymer}{?}
\draw[line](nucleicAcidText) to (nucleicAcidTextSecond);
\draw[line, dashed, brown](nucleicAcid) to (nucleicAcidText);
\question{right=of nucleicAcidText}{nucleotide}{Nucleotide}{?}
\comment{below=0 of nucleotide}{nucleotideTextFirst}{\textcolor{black}{1. }Provide Chemical Energy}{\textcolor{black}{1. }?}
\comment{below=0 of nucleotideTextFirst}{nucleotideTextSecond}{\textcolor{black}{2. }Participate in Cell Signaling}{\textcolor{black}{2. }?}
\comment{above=of nucleotide}{nucleotideText}{Monomer}{?}
\draw[line](nucleotide) to (nucleotideText);
\node at($(nucleicAcidText.east)!.5!(nucleotide.west)$){\huge$\ni$};
\comment{right=of nucleotide}{nucleotideStructure}
{
\chemfig{
O^{-}-P(=[:90]O)(-[:-90]O^{-})-O
-[:-30]-[:-90]([:-18]<[:-60](-[6]OH)-[0,,,,line width=3pt]>[:60]
(-[:90]N
*5([:72]-
*6(-N=-N=(-NH_2)-=)
--N=-))
-[:150,1.15]O-[:210,1.15])
}}{?}
\draw[line](nucleotide) to (nucleotideStructure);
\comment{right=0 of nucleotideStructure}{nucleotideStructureTextSecond}
{\textcolor{black}{Center: }Ribose \textcolor{black}{or} \textcolor{blue}{Deoxyribose}}{\textcolor{black}{Center: }?}
\comment{above=of nucleotideStructureTextSecond.west, anchor=west}{nucleotideStructureTextFirst}
{\textcolor{black}{Upper Left: }Phosphate Group}{\textcolor{black}{Upper Left: }?}
\comment{below=of nucleotideStructureTextSecond.west, anchor=west}{nucleotideStructureTextThird}
{\textcolor{black}{Upper Right: }Nucleobase}{\textcolor{black}{Upper Right: }?}

\draw[dashed, brown](nucleotideStructureTextSecond.north)
-| (nucleotideStructureTextThird.south west)
-| (nucleotideStructureTextSecond.east)
|- coordinate(nucleoside) cycle;
\question{right=of nucleoside}{nucleosideText}{Nucleoside}{?}
\draw[dashed, brown](nucleoside) to (nucleosideText);

\question{above=of transcription}{polymerase}{RNA Polymerase}{?}
\draw[line](transcription) to (polymerase);
\question{above=of polymerase}{transcriptionFactor}{Transcription Factor}{?}
\draw[line](polymerase) to node[circle, fill=white]{+} (transcriptionFactor);
\comment{right=of polymerase}{promotor}{Bind to the Promotor}{?}
\draw[arrow](polymerase) to (promotor);
\comment{right=of promotor}{transcriptionBubble}{Generate a Transcription Bubble}{?}
\draw[arrow](promotor) to (transcriptionBubble);
\comment{above=of transcriptionBubble}{transcriptionBubbleText}{\textcolor{black}{Size: }$12\sim 14$ base pairs}{\textcolor{black}{Size: }?}
\draw[line](transcriptionBubble) to (transcriptionBubbleText);
\coordinate(base) at($(nucleotideStructureTextThird.south)+(-1,-.5)$);
\question {below=0 of base}{guanine}{Guanine}{?}
\question{left=.5 of guanine}{adenine}{Adenine}{?}
\question{right=.5 of guanine}{cytosine}{Cytosine}{?}
\question{right=.5 of cytosine}{thymine}{Thymine}{?}
\comment{right=of thymine}{uracil}{Replaced with Uracil in RNA}{?}
\draw[line](thymine) to (uracil);
\draw[arrow, latex-latex, out=300, in=240](guanine) to node[below]{pair} (cytosine);
\draw[arrow, latex-latex, out=315, in=225](adenine) to (thymine);
% \comment{above=of nucleotideStructureTextFirst}{phosphateText}{Nucleotides are connected by the phosphate molecule and form the long strand of DNA}{?}
% \draw[line](nucleotideStructureTextFirst) to (phosphateText);
\draw(nucleotideStructureTextFirst.east) -- ++(.3,0) coordinate(connection) |- (nucleotideStructureTextSecond.east);
\comment{above right=.5 of connection}{connectionText}{Connected to form the long stand of DNA}{?}
\draw[-](connection) to (connectionText.south west);

https://www.vedantu.com/question-answer/which-of-the-following-would-not-be-considered-class-12-biology-cbse-5fb37455b8881b0c51d41bf3
https://www.youtube.com/watch?v=ad6h4ErsSTY

