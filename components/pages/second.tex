\newpage
\section*{tRNA}
While the specific nucleotide sequence of an mRNA specifies which amino acids are incorporated into the protein product of the gene from which the mRNA is transcribed, \hl{the role of tRNA is to specify which sequence from the genetic code corresponds to which amino acid.} The mRNA encodes a protein as a series of contiguous codons, each of which is recognized by a particular tRNA. One end of the tRNA matches the genetic code in a three-nucleotide sequence called the anticodon. The anticodon forms three complementary base pairs with a codon in mRNA during protein biosynthesis.

On the other end of the tRNA is a covalent attachment to the amino acid that corresponds to the anticodon sequence. Each type of tRNA molecule can be attached to only one type of amino acid, so each organism has many types of tRNA. Because the genetic code contains multiple codons that specify the same amino acid, there are several tRNA molecules bearing different anticodons which carry the same amino acid.

The covalent attachment to the tRNA $3^{\prime}$ end is catalysed by enzymes called aminoacyl tRNA synthetases. During protein synthesis, tRNAs with attached amino acids are delivered to the ribosome by proteins called elongation factors, which aid in association of the tRNA with the ribosome, synthesis of the new polypeptide, and translocation (movement) of the ribosome along the mRNA. If the tRNA's anticodon matches the mRNA, another tRNA already bound to the ribosome transfers the growing polypeptide chain from its $3^{\prime}$ end to the amino acid attached to the $3^{\prime}$ end of the newly delivered tRNA, a reaction catalysed by the ribosome. A large number of the individual nucleotides in a tRNA molecule may be chemically modified, often by methylation or deamidation. These unusual bases sometimes affect the tRNA's interaction with ribosomes and sometimes occur in the anticodon to alter base-pairing properties.
\\\\src: \url{https://en.wikipedia.org/wiki/Transfer_RNA}