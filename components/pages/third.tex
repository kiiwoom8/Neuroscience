\newpage
\section*{rRNA}
Universally conserved secondary structural elements in rRNA among different species show that these sequences are some of the oldest discovered. They serve critical roles in forming the catalytic sites of translation of mRNA. \hl{During translation of mRNA, rRNA functions to bind both mRNA and tRNA to facilitate the process of translating mRNA's codon sequence into amino acids.} rRNA initiates the catalysis of protein synthesis when tRNA is sandwiched between the SSU and LSU. In the SSU, the mRNA interacts with the anticodons of the tRNA. In the LSU, the amino acid acceptor stem of the tRNA interacts with the LSU rRNA. The ribosome catalyzes ester-amide exchange, transferring the C-terminus of a nascent peptide from a tRNA to the amine of an amino acid. These processes are able to occur due to sites within the ribosome in which these molecules can bind, formed by the rRNA stem-loops. A ribosome has three of these binding sites called the A, P and E sites:
\\

\textbullet \space In general, the A (aminoacyl) site contains an aminoacyl-tRNA (a tRNA esterified to an amino acid on the 3' end).

\textbullet \space The P (peptidyl) site contains a tRNA esterified to the nascent peptide. The free amino (NH2) group of the A site tRNA attacks the ester linkage of P site tRNA, causing transfer of the nascent peptide to the amino acid in the A site. This reaction is takes place in the peptidyl transferase center

\textbullet \space The E (exit) site contains a tRNA that has been discharged, with a free 3' end (with no amino acid or nascent peptide).
\\

A single mRNA can be translated simultaneously by multiple ribosomes. This is called a polysome.

In prokaryotes, much work has been done to further identify the importance of rRNA in translation of mRNA. For example, it has been found that the A site consists primarily of 16S rRNA. Apart from various protein elements that interact with tRNA at this site, it is hypothesized that if these proteins were removed without altering ribosomal structure, the site would continue to function normally. In the P site, through the observation of crystal structures it has been shown the 3' end of 16s rRNA can fold into the site as if a molecule of mRNA. This results in intermolecular interactions that stabilize the subunits. Similarly, like the A site, the P site primarily contains rRNA with few proteins. The peptidyl transferase center, for example, is formed by nucleotides from the 23S rRNA subunit. In fact, studies have shown that the peptidyl transferase center contains no proteins, and is entirely initiated by the presence of rRNA. Unlike the A and P sites, the E site contains more proteins. Because proteins are not essential for the functioning of the A and P sites, the E site molecular composition shows that it is perhaps evolved later. In primitive ribosomes, it is likely that tRNAs exited from the P site. Additionally, it has been shown that E-site tRNA bind with both the 16S and 23S rRNA subunits.